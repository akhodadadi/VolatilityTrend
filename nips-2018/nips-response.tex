\documentclass[11pt]{article}
% Include statements
\usepackage{graphicx}
\usepackage{amsfonts,amssymb,amsmath,amsthm}
\usepackage[numbers,square]{natbib}
\usepackage[left=1in,top=1in,right=1in,bottom=1in,nohead]{geometry}
\usepackage{pdfsync}
\usepackage{hyperref}
\hypersetup{backref,colorlinks=true,citecolor=blue,linkcolor=blue,urlcolor=blue}
\renewcommand{\qedsymbol}{$\blacksquare$}

% Bibliography
\bibliographystyle{mybibsty}

% Theorem environments
\newcommand{\email}[1]{\href{mailto:#1}{#1}}

\title{Author response: NIPS 2018, Paper 3656}
\author{Modeling trend in temperature volatility using generalized
  LASSO}
\date{\today}


\begin{document}
\noindent {\bf Author response: NIPS 2018, Paper 3656\\}
Modeling trend in temperature volatility using generalized
  LASSO\\
\today

\hspace{10pt}

We wish to thank the reviewers for their careful and well-informed
comments. We are especially grateful for the literature suggestions
given by reviewer 1. The Gibberd et al.\ (2017) and Monti et al.\ (2014) papers especially deserve
mention in our paper, and neither was familier to us. Both take a similar approach to ours, and while
different in a number of key ways, deserve a more careful discussion
than can be provided here. In particular, we should, at minimum, try
their algorithms or give reasons why we cannot. However, we feel these papers provide
a good launching point to discuss the main
issue, mentioned by all three reviewers.

\paragraph{Main issue} All three reviewers felt that this paper lacked sufficient novelty
relative to existing literature. The novelty of our paper is best understood as an attempt to
apply a slightly modified $\ell_1$-trend filter to a very large and
very important dataset. It is certainly true that $\ell_1$
regularization is everywhere, as is trend filtering, and variance
estimation under such a penalty. In that sense, our paper is not
particularly new. However, we found that the size of our problem
(coupled with a non-quadratic likelihood function) meant that standard
methods were computationally infeasible. The time-varying networks
used in Monti, Hallac et al.\ (2017), and Gibberd, have on the order of hundreds of
time points and dozens to hundreds of nodes. Our data have (see lines
100--105) 3500 time points and ~25,000 nodes. It is this massive
increase in throughput which makes our methods both relevant and
novel. Even building the penalty matrix (or loading the data) is
relatively infeasible. This is also the reason that our simulation
size is relatively paltry: to enable many replications performed
quickly. Note however that Gibberd uses 10 nodes and 50 timepoints in
their simulations, so our example with 35 nodes and 780 timepoints is
not that out of the ordinary.

\paragraph{Why variance} Finally, there was some confusion as to why the variance would be
important here (rather than the mean). This is for a number of reasons
which we tried to articulate carefully in the introduction.
\begin{enumerate}
\item Instrument bias in the satellite increases over time so
  examining the mean over time conflates that bias with any actual
  change in mean (though the variance is unaffected).
\item Extreme weather events (hurricanes, droughts, wildfires in
  California, heatwaves in Europe) are driven more strongly by
  increases in variance than by increases in mean.
\item Even if the global mean temperature is constant, there may still
  be climate change. In fact, atmospheric physics suggests that,
  across space, average temperatures should not change (extreme cold
  in one location is offset by heat in another). But if swings across
  space are becoming more rapid, then even with no change in mean global
  temperature over time, the variance is increasing, again leading to
  increases in extreme events.
\end{enumerate}

\paragraph{Minor issues} We will of course be careful to address typos
(thank you for pointing them out!) as well as attempt to clarify
points of the paper which were less precise than they should have
been. And, really, thank you for your reviews. 








\bibliography{AllReferences.bib}
\end{document}

